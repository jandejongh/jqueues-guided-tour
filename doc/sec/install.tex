\section{The \texttt{jsimulation} and \texttt{jqueues} Libraries}

In order to use \lstinline|jsimulation|
  and \lstinline|jqueues|, you have to install
  them first,
  which requires an Internet connection.
The first public releases of \lstinline|jqueues| and
  \lstinline|jsimulation| have version number \lstinline|5.2.0|;
  they have been released under the Apache v2.0 license.
From that version number onward,
  both libraries are distributed as \lstinline|Maven|
  projects
  available from \lstinline|github.com|
  and the Maven Central Repository
  (whichever suits you).
  
Since both \lstinline|jsimulation|
  and \lstinline|jqueues| are libraries
  and hardly support stand-alone operation,
  we assume that you intend to install
  them both as dependencies to your own project.
You have several options, but the two most obvious ones are:
\begin{itemize}
	\item Install the libraries from \lstinline|github|,
	        open them as Maven {\em projects\/} in your IDE
	        and add them as dependencies to your own project.
	      If you use Maven yourself for the latter,
	        you only have to add
	        the dependency on \lstinline|jqueues| in the
	        \lstinline|pom.xml|.
	      (You do not have to add \lstinline|jsimulation|
            because Maven does this automatically for you.)
	\item Create your own Maven project and add
	        \lstinline|jqueues| as a dependency,
	        taken from the Maven Central Repository.
\end{itemize}
In both cases,
  you will need \lstinline|maven| installed and properly configured on your system.
It is also highly recommended to install
  \lstinline|maven| support in your IDE,
  so that it can directly open \lstinline|maven| projects.

In the first case, you need \lstinline|git| as well,
  and you should clone both libraries
  from \lstinline|github| as shown below:
\begin{itemize}
	\item \lstinline|$ git clone https://www.github.com/jandejongh/jsimulation|
	\item \lstinline|$ git clone https://www.github.com/jandejongh/jqueues|
\end{itemize}
Note that \lstinline|jsimulation| and \lstinline|jqueues|
  can only be built against \lstinline|Java 1.8| and higher.

In the second case,
  add the XML fragment shown in Listing \ref{jqueues_pom_dependency}
  to the dependencies
  section in your \lstinline|pom.xml|.
Please make sure that you double-check the version number in the XML file\footnote{
You may want to verify the latest stable release number
from either \lstinline|github| or Maven Central.
This Guided Tour applies to release 5.2.0 and beyond.}. 
\begin{lstfloat}
\begin{lstlisting}[
	caption={The \texttt{dependency} section for \texttt{jqueues} in a \texttt{pom.xml}.},
	label=jqueues_pom_dependency,
	basicstyle=\tiny]
<dependency>
	<groupId>org.javades</groupId>
	<artifactId>jqueues</artifactId>
	<version>5.2.0</version>
	<scope>compile</scope>
	<type>jar</type>
</dependency>
\end{lstlisting}
\end{lstfloat}
The second case is safer as it uses stable, frozen, versions
  of the libraries released to Maven Central.
These releases are signed and cannot be changed without
  increasing the version number.
  
\section{Version Numbering}

For both libraries, we use three-level version numbering:
\begin{itemize}
	\item The third, lowest, level is reserved for bug fixes,
	        \lstinline|javadoc| improvements
	        and code (layout) "beautifications".
	\item The second, middle, level is reserved for functional extensions
	        that do not break existing code (with the same major version number).
	      Think of adding another queue, job or listener type.
	\item The third, major, level is reserved for changes to the core
	        interfaces and classes that are likely to break existing code.
\end{itemize}
This implies that you can (should be able to) always "upgrade" to a later version
  from Maven Central
  as long as the major number remains the same.
Upgrading from \lstinline|github.com| requires a bit of care,
  as the latest version may not be stable yet.
  
Despite the fact that we take utmost efforts
  to {\em not\/} break existing code
  with upgrades of middle and minor version numbers,
  we cannot always avoid this.
For instance,
  we may realize that a method should be
  \lstinline|final| or \lstinline|private|
  and attempt to fix that in an apparent innocent update,
  but you may have overridden (or used)
  that particular method already in your code
  to suit your own purposes.
Needless to say,
  we did not expect you to override (or just use) that particular method in your code,
  just as well as you did not expect that you were not supposed to do so.
But in the end,
  your code may not be compile-able after the upgrade.
In order to avoid this, we recommend that you
\begin{itemize}
  \item Prefer interface methods rather than specific ones from classes,
        since the chance that we consider updates of the interface
        as being "minor" is virtually nil.
  \item Only override methods for which the \lstinline|javadoc| explicitly
        states that they are intended to be overridden.
\end{itemize}

\section{The \texttt{jqueues-guided-tour} Project}

All example code shown in this document is available
  from the \lstinline|jqueues-guided-tour|
  project on \lstinline|github|.
The code is organized as a Maven project.
In addition to the example code,
  it also contains all the source 
  files (\LaTeX\ and other)
  to the present document.
Bear in mind, though, that the documentation and
  example code in \lstinline|jqueues-guided-tour|
  are both released under a more restrictive license than
  \lstinline|jsimulation| and \lstinline|jqueues|.
In short, you are allowed to use the documentation and
  example code to whatever purpose.
You may also redistribute both in unmodified form.
However,
  redistributing {\em modified\/} versions
  of either or both of them
  requires the explicit permission from
  the legal copyright holder.