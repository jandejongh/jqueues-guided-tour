This chapter describes the event and event-list features
  that are available from the \lstinline{jsimulation} package.
Note that \lstinline{jsimulation} is a dependency of \lstinline{jqueues}.
In most usage scenarios,
  there is no need to directly manipulate events or the event-list;
  the preferred method is to use {\em utility\/} methods for that.
However,
  in order to describe in more detail the models of
  entities, jobs and queues,
  a basic understanding of what goes on under the hood
  of a \lstinline|DefaultSimEventList|
  is very helpful.

\section{Creating the Event List and Events}

At the very heart of every simulation experiment
  in \lstinline{jqueues}
  is the so-called {\em event list}.
The event list obviously holds the events,
  keeps them ordered,
  and maintains a notion of "where we are" in a simulation run.
Together, an event list and the events it contains define
  the precise sequence of actions taken in a simulation.
The code snipplet in Listing \ref{EventList1_main}
  shows how to create an event list and
  schedule two (empty) events, one at $t_{1}=5.0$ and one at $t_{2}=10$,
  and print the resulting event list on \lstinline{System.out}.
In \lstinline{jsimulation},
  the event list is of type \lstinline{SimEventList};
  events are of type \lstinline{SimEvent},
  respectively.
Since both of them are Java {\em interfaces}, you need implementing classes
  to instantiate them: \lstinline{DefaultSimEventList} for an event list;
  \lstinline{DefaulSimEvent} for an event;
  typically you need a single event list and numerous events.
  
\begin{lstfloat}
\begin{lstlisting}[
    caption={Creating the event list and populating it with events.},
    label=EventList1_main,
    basicstyle=\tiny]
	
  final SimEventList el = new DefaultSimEventList (-5);
  final SimEvent e1 = new DefaultSimEvent (5.0);
  final SimEvent e2 = new DefaultSimEvent (10.0);
  el.add (e1);
  el.add (e2);
  el.print ();
  el.run ();
  System.out.println ("Finished!");
	
\end{lstlisting}
\end{lstfloat}

As explained in the previous chapter,
  the \lstinline|double| argument in the \lstinline|DefaultSimEventList|
  constructor is the initial time on the event list,
  its so-called {\em default reset time}.
The \lstinline{double} argument in the \lstinline{DefaultSimEvent} constructor
  (of which there are several)
  is the {\em schedule time\/} of the event on the event list.
Events, once created,
  are scheduled on the event list through the \lstinline|add| method;
  the event list stores the events until use
  and maintains the proper order between them.
The output of the code snipplet is shown in Listing \ref{EventList1_out}\footnote{
  We may have improved the layout in the meantime.}:

\begin{lstfloat}
	\begin{lstlisting}[
	caption={Output of Listing \ref{EventList1_main}.},
	label=EventList1_out,
	basicstyle=\tiny]
	
	SimEventList EventList[t=-5.0], class=DefaultSimEventList, time=-5.0:
	t=5.0, name=No Name, object=null, action=null.
	t=10.0, name=No Name, object=null, action=null.
	Finished!
	
	\end{lstlisting}
\end{lstfloat}

By virtue of the call to \lstinline|el.print|,
  the output shows the name of the event list
  (as obtained from its \lstinline{toString} method)
  and the current time ($-5$) in the first row,
  and then the events in the list in the proper order.
Beware that the event-list is printed
  before the \lstinline|el.run| statement;
  it would be empty afterwards.

Perhaps surprisingly,
  in \lstinline{jsimulation},
  the schedule time is actually held on the event,
 {\em not\/} on the event list.
Also, a \lstinline{SimEventList} is inheriting from \lstinline{SortedSet}
  from the Java Collections Framework.
These choices have the following consequences:
\begin{itemize}
  \item Each \lstinline{SimEvent} can be present {\em at most once\/}
        in a \lstinline{SimEventList}.
        You cannot reuse a single event instance
          (like a job creation and arrival event)
          by scheduling it multiple times on the event list.
        Instead, you must either use separate event instances,
          or reschedule the event the moment it leaves the event list.
  \item You cannot (more precisely, {\em should not\/})
          modify the time on the event while it is
          scheduled on an event list.
  \item You always have access to the (intended) schedule time of the event,
          without having to refer to an event list
          (if the event is scheduled at all)
          or use a separate variable
          to keep and maintain that time.
  \item The events must be equipped with a {\em total ordering\/}
          (imposed by \lstinline{SortedSet})
          and distinct events should not be equal (imposed by us).
          This means that for each pair of (distinct) events scheduled
          on a \lstinline{SimEventList},
          one of them is always strictly larger than the other
          (in the ordering, they cannot be "equal").
  \item If two or more events with identical schedule times are
          scheduled on a single event list,
          their relative order needs to be determined by other means
          than their schedule time.
        The \lstinline|DefaultSimEventList| uses a
          random-number generator to break such ties.
        If, for some reason,
          you want to maintain {\em insertion order\/},
          please have a look at \lstinline|DefaultSimEventList_IOEL|.
        Note that \lstinline|IOEL| stands for Insertion Order Event List.
        But be warned: all (concrete) queue types in \lstinline{jqueues}
          are specified against random ordering
          of simultaneous events.
\end{itemize}

Clearly, there is a lot more to say about simultaneous events,
  and about the reasons we chose for their random ordering
  while processing them, but we defer a detailed discussion for a later section.
Nonetheless, it is important to realize
  that while an event say \lstinline|e1|
  is being processed at some time $t$,
  any other event say \lstinline|e2|
  scheduled at the same time on the event list
  is {\em always\/} processed after completion of \lstinline|e1|.
Even if \lstinline|e1| itself actually schedules \lstinline|e2|. 
In other words,
  \lstinline|jsimulation| does {\em not\/} support the concept
  of {\em event preemption\/},
  and the action of an event (see below)
  is always processed atomically.
This implies that it will not work
  to use the event list
  (1) to get something done "immediately after" the completion of an event,
  (2) to do something "when all other events at $t$" are done",
  and
  (3) to process an event \lstinline|e2|
     while processing an event \lstinline|e1|
     and then returning to the original event \lstinline|e1|.

\section{Events}

The output in Listing \ref{EventList1_out}
  shows four properties of a \lstinline|SimEvent|:
\begin{itemize}
\item \lstinline|Time|:
        The (intended) schedule time of the event (default $-\infty$).
\item \lstinline|Name|:
        The name of the event,
        which is only used for logging and output (default "No Name").
\item \lstinline|Object|:
        A general-purpose object available for storing information
        associated with the event
        (\lstinline{jsimulation} nor \lstinline{jqueues}
        uses this field;
        its default value is \lstinline{null}).
\item \lstinline|EventAction|:
        The action to take, a \lstinline{SimEventAction}
        (default \lstinline{null}),
        described in the next section.
\end{itemize}
Each property has corresponding getter and setter methods
  on every \lstinline|SimEvent|.
In addition,
  \lstinline|DefaultSimEvent| features
  multiple constructors that allow
  direct setting all or some of these properties upon construction.

\section{Actions}

A \lstinline{SimEventAction} defined what needs to be done by the time an event
  is {\em executed\/} or {\em processed}.
In Java terms, a \lstinline{SimEventAction} is an interface with
  a single abstract method which is invoked when the event is processed,
  in other words, it is a \lstinline|FunctionalInterface|
  that can be used in lambda expressions.
We show its declaration in Listing \ref{SimEventAction}.

\begin{lstfloat}
\begin{lstlisting}[
  caption={The \texttt{SimEventAction} interface.},
  label=SimEventAction,
  basicstyle=\tiny]

  @FunctionalInterface
  public interface SimEventAction<T>
  {

    /** Invokes the action for supplied {@link SimEvent}.
     *
     * @param event The event.
     *
     * @throws IllegalArgumentException If <code>event</code> is <code>null</code>.
     * 
     */
    public void action (SimEvent<T> event);

  }

\end{lstlisting}
\end{lstfloat}

There are several ways to create a \lstinline|SimEventAction|
  but nowadays, by far the easiest is to use lambda expressions,
  as shown in Listing \ref{SimEventAction_main}.
Note that we are now using the full \lstinline{DefaultSimEvent} constructor,
  passing a name, and supplying a \lstinline{SimEventAction}
  through a lambda expression.
The generated output is shown in Listing \ref{SimEventAction_out}.
Note that we replaced the package and class identification
  of the action with \lstinline|X| for formatting purposes.

\begin{lstfloat}  
\begin{lstlisting}[
  caption={Creating and using \texttt{SimEventAction}s.},
  label=SimEventAction_main,
  basicstyle=\tiny]
  
  final SimEventList el = new DefaultSimEventList (0);
  final SimEvent e = new DefaultSimEvent ("My First Real Event", 5.0, null, ((event) ->
    {
      System.out.println ("Event=" + event + ", time=" + event.getTime () + ".");
    }));
  el.add (e);
  el.print ();
  el.run ();
  el.print ();

\end{lstlisting}
\end{lstfloat}

\begin{lstfloat}
\begin{lstlisting}[
  caption={Example output of Listing \ref{SimEventAction_main}.},
  label=SimEventAction_out,
  basicstyle=\tiny]

  SimEventList EventList[t=0.0], class=DefaultSimEventList, time=0.0:
    t=5.0, name=My First Real Event, object=null, action=X$$Lambda$1/1826771953@65ab7765.
  Event=My First Real Event, time=5.0.
  SimEventList EventList[t=5.0], class=DefaultSimEventList, time=5.0:
    EMPTY!

\end{lstlisting}
\end{lstfloat}

\section{Processing the Event List}

Once the events of your liking are scheduled on the event list,
  you can start the simulation by {\em processing\/} or {\em running\/}
  the event lists.
Processing the event list will cause the event list to
  equentially invoke the actions attached to the events
  in increasing-time order.
There are several ways to process a \lstinline{SimEventList}:
\begin{itemize}
  \item You can process the event list until it is empty with the \lstinline{run} method.
  \item You can process the event list until some specified (simulation) time with the
          \lstinline{runUtil} method.
  \item You can {\em single-step\/} through the event list with the
          \lstinline{runSingleStep} method.
\end{itemize}
You can check whether an event list is being processed through its \lstinline{isRunning}
  method.

While processing, the event list maintains a {\em clock}
  holding the (simulation) time of the current event.
You can get the time from the event list through \lstinline{getTime} nethod,
  although you can obtain it more easily from the event itself.
You can insert new events while it is being processed,
  {\em but these events must not be in the past}.
Once the event list detects insertion of events in the past,
  it will throw and exception.

Note that processing the event list
  is thread-safe in the sense that all methods involved
  need to obtain a {\em lock} before being able to process the list.
Trying to process an event list that is already being processed
  from another thread,
  or from the thread that currently processes the list,
  will lead to an exception.
Note that currently there is no safe, atomic, way
  to process an event list on the condition that is
  is not being processed already.
Though you can check with \lstinline{isRunning}
  whether the list is being processed or not,
  the answer from this method has zero validity lifetime.

\section{Utility Methods for Scheduling Events}

A \lstinline{SimEventList} supports various methods for
  directly scheduling events and actions
  without the need to generate both
  the \lstinline{SimEvent} {\em and\/} the \lstinline{SimEventAction}.
In most cases, the availability of one of the objects suffices.
In Table \ref{SimEventList_Scheduling}
  we show the most common utility methods for scheduling on a \lstinline{SimEventList}.
The use of these utility methods is highly preferred over
  direct manipulation of the underlying \lstinline|SortedSet| interface,
  because we (may) intend to delete the \lstinline|SortedSet|
  dependency in future releases altogether.

\begin{table}[h]
\caption{Utility methods for scheduling on a \lstinline|SimEventList|.}
\label{SimEventList_Scheduling}
\begin{center}
\begin{tabular}{|l|}
\hline
\multicolumn{1}{|c|}{\bf Utility methods for scheduling on \lstinline|SimEventList|} \\
\hline
\lstinline[basicstyle=\footnotesize]!void schedule (E)! \\
Schedules the event at its own time\footnote{
Release 5.3.0 and higher only.}.\\
\hline
\lstinline[basicstyle=\footnotesize]!boolean cancel (E)! \\
Cancels (removes) a scheduled event, if present. \\
\hline
\lstinline[basicstyle=\footnotesize]!void schedule (double, E)! \\
Schedules the event at given time.\\
\hline
\lstinline[basicstyle=\footnotesize]!reschedule (double, E)! \\
Reschedules (if present, else schedules) the event at given new time.\\
\hline
\lstinline[basicstyle=\footnotesize]!E schedule (double, SimEventAction, String)! \\
Schedules the action at given time with given event name.\\
\hline
\lstinline[basicstyle=\footnotesize]!void scheduleNow (E)! \\
Schedules the event now.\\
\hline
\lstinline[basicstyle=\footnotesize]!E schedule (double, SimEventAction)! \\
Schedules the action at given time with default event name.\\
\hline
\lstinline[basicstyle=\footnotesize]!E scheduleNow (SimEventAction, String)! \\
Schedules the action now with given event name.\\
\hline
\lstinline[basicstyle=\footnotesize]!E scheduleNow (SimEventAction)! \\
Schedules the action now with default event name.\\
\hline
\end{tabular}
\end{center}
\end{table}

Note that \lstinline{E} refers to the so-called {\em generic-type argument\/}
  of \lstinline{SimEventList}.
The prototype is \lstinline!SimEventList<E extends SimEvent>!.
The use of the generic type \lstinline|E|
  allows you to restrict the use of a \lstinline|SimEventList|
  to certain types of \lstinline|SimEvent|s,
  but for now \lstinline!E! can be simply read as a \lstinline{SimEvent}.

For any of the utilty methods that take a \lstinline{SimEventAction}
  as argument, a new \lstinline{SimEvent} is created on the fly,
  and returned from the method.
Upon return from these methods,
  the newly created event has already been scheduled,
  and you {\em really\/} should not schedule it again.

So, how to {\em remove\/} events and actions from the event list?
Well, since \lstinline{SimEventList} implements the \lstinline{Set} interface for
  \lstinline{SimEvent} members, removing an event \lstinline{e}
  from an event list \lstinline{el} is as simple as
  \lstinline{el.remove (e)}.
However,
  the preferred method is \lstinline|el.cancel (e)|.

\section{Summary}

The fundamental concepts in
\lstinline|jsimulation| are:
\begin{itemize}
\item The \lstinline|Java| package named \lstinline|jsimulation|
  is a library for (single-threaded) discrete-event simulation.
\item The \lstinline|Java| package named \lstinline|jqueues|
  is a library for (single-threaded) discrete-event simulation
  of queueing systems.
  The library depends on \lstinline|jsimulation|.
\item In order to perform discrete event simulations,
  an event list is needed, on which events can be scheduled.
  The event list maintains an ordering of the events it contains
  in non-decreasing simulation time.
  Typically, a single instance of an event list is used
  throughout the entire simulation study.
  The corresponding (abstract) types
  for event lists and events are defined in \lstinline|jsimulation|,
  and named \lstinline|SimEventList| and \lstinline|SimEvent|, respectively.
  This package also provides a reasonable implementation for
  a \lstinline|SimEventList| named \lstinline|DefaultSimEventList|.
\item On a \lstinline|SimEventList|, all scheduled \lstinline|SimEvent|s are
  unique; you cannot schedule a \lstinline|SimEvent| more than once
  on a single \lstinline|SimEventList|.
  Typically, \lstinline|SimEvent|s are created and scheduled through
  various utility methods.
\item The time at which a \lstinline|SimEvent| is scheduled,
  is kept on the \lstinline|SimEvent| itself,
  and available though the \lstinline|SimEvent.getTime| method.
  Once scheduled, you cannot change the time of a \lstinline|SimEvent|.
  You can, however, reschedule it at a different time
  through the \lstinline|SimEventList.reschedule| method.
\item It is perfectly legel if multiple \lstinline|SimEvent|s are scheduled
  at the same time.
  On a \lstinline|DefaultSimEventList|, they are processed in random order.
\item With each \lstinline|SimEvent|, an action is associated that determines
  what to do when the event is processed by the event list.
  The generic type in \lstinline|jsimulation| is \lstinline|SimEventAction|.
  Unlike \lstinline|SimEvent|s, \lstinline|SimAction| need not be unique
  on the event list, and can be shared among different events.
\item Once sufficient events have been scheduled,
  a simulation experiment starts by running or processing
  the event list.
  In \lstinline|jsimulation|,
  you can run the \lstinline|SimEventList| until it is exhausted
  of events through the \lstinline|SimEventList.run| method,
  until it has reached a specific simulation time
  through the \lstinline|SimEventList.runUntil| method,
  or on an event-by-event basis through \lstinline|SimEventList.runSingleStep|.
\item A \lstinline|SimEventList| keeps a notion of simulation time.
  It is available through \lstinline|SimEventList.getTime|.
  While running,
  this is always the scheduled time of the current event being processed.
  When not, it is always smaller than or equal to the time
  on the first scheduled \lstinline|SimEvent|.
\item You cannot schedule (at the risk of an \lstinline|Exception|)
  a \lstinline|SimEvent| with time
  strictly smaller than the current simulation time
  on the \lstinline|SimEventList|.
\item Event may be scheduled simultaneously,
  in which case their order of processing is {\em random}.
\item Events may be scheduled at $t=-\infty$ and $t=+\infty$.
\item The \lstinline|SimEventList.reset|
  and \lstinline|SimEventList.reset (double)|
  methods reset the event list,
  meaning all scheduled \lstinline|SimEvent|s are removed from the list,
  and the time on the event list is set to its default time (first method)
  or given time.
  The typical use case of these methods is running the simulation again
  (for instance, for variance-reduction purposes).
  You cannot invoke either method while the event list is being processed,
  at the risk of an \lstinline|Exception|.
\end{itemize}