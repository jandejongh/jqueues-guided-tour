In \lstinline|jqueues|, monitoring the progress of a running simulation,
  or perhaps calculating statistiscs on it,
  starts with chosing the proper {\em listeners}.
During the simulation,
  queues and jobs, from hereon collectively referred to as {\em entities},
  are obliged to notify registered listeners of (at least) {\em all\/}
  changes to their states.
A listener is a \lstinline|Java| object implementing the required methods
  allowing such notifications from the entity.

Since in most practical simulation studies,
the ambition level is somewhat higher that showing events on \lstinline|System.out|,
we will delve somewhat deeper into listeners types and how
to create, modify and register them.

In the example of Listing \ref{simExample1_main},
  we used a convenience method \lstinline|registerStdOutSimQueueListener|
  to register a listener at the \lstinline|FCFS_B| queue
  that simply writes the details of such notifications
  to the standard output, \lstinline|Sytem.out| in \lstinline|Java|.
This is extremely handy for initial testing of a simulation,
  but in almost all cases,
  a more sophisticated listener is required;
  one you have to create yourself.
Luckily, \lstinline|jqueues| comes with a large collection of
  listener implementations, each for a specific purpose, that
  you can modify to suit your needs.

Restricting ourselves for the moment to queue listeners,
  we can create and register
  our own listener that reports to \lstinline|System.out|
  in the code in Listing \ref{simExample1_main}:

\begin{lstfloat}
\begin{lstlisting}[
  caption={Creating and registering a listener.},
  label=simExample1_listener1,
  basicstyle=\tiny]

  final SimQueueListener listener = new StdOutSimQueueListener ();
  queue.registerSimEntityListener (listener);

\end{lstlisting}
\end{lstfloat}

Running the program of \ref{simExample1_main} again with
  this modified code fragment will yield (roughly)
  the same output, so we have not gained anything so far.
However,
  a \lstinline|StdOutSimQueueListener| allows
  all (notification) methods to be overridden,
  so we can, for instance,
  suppress certain notifications in the output like this:

\begin{lstfloat}
\begin{lstlisting}[
  caption={Suppressing specific notifications in a \texttt{StdOutSimQueueListener}.},
  label=simExample1_listener1_suppress,
  basicstyle=\tiny]

    final SimQueueListener listener = new StdOutSimQueueListener ()
    {
      
      @Override
      public void notifyStateChanged (double time, SimEntity entity, List notifications) {}

      @Override
      public void notifyUpdate (double time, SimEntity entity) {}
      
      @Override
      public void notifyStartQueueAccessVacation (double time, SimQueue queue) {}

      @Override
      public void notifyStopQueueAccessVacation (double time, SimQueue queue) {}

      @Override
      public void notifyNewStartArmed (double time, SimQueue queue, boolean startArmed) {}
      
      @Override
      public void notifyOutOfServerAccessCredits (double time, SimQueue queue) {}

      @Override
      public void notifyRegainedServerAccessCredits (double time, SimQueue queue) {}

    };
    queue.registerSimEntityListener (listener);

\end{lstlisting}
\end{lstfloat}

In Listing \ref{simExample1_listener1_suppress},
  we modify the \lstinline|StdOutSimQueueListener| by overriding
  the notification methods for
  server-access credits
  and queue-access vacations
  (which we do have not described yet),
  for the \lstinline|StartArmed|-related notifications
  and replacing them with empty methods,
  effectively suppressing their respective
  output on \lstinline|System.out|.
In addition,
  we suppress the \lstinline|UPDATE| and \lstinline|STATE CHANGED|
  notifications.
Our modified listener yields the following output:

\begin{lstfloat}
\begin{lstlisting}[
  caption={Example output of Listing \ref{simExample1_main} with the modified listener of Listing
  \ref{simExample1_listener1_suppress}},
  label=simExample1_listener1_out,
  basicstyle=\tiny]

 t=0.0, queue=FCFS_B[2]: ARRIVAL of job 0.
 t=0.0, queue=FCFS_B[2]: START of job 0.
 t=0.0, queue=FCFS_B[2]: DEPARTURE of job 0.
 t=1.0, queue=FCFS_B[2]: ARRIVAL of job 1.
 t=1.0, queue=FCFS_B[2]: START of job 1.
 t=2.0, queue=FCFS_B[2]: ARRIVAL of job 2.
 t=3.0, queue=FCFS_B[2]: ARRIVAL of job 3.
 t=3.2, queue=FCFS_B[2]: DEPARTURE of job 1.
 t=3.2, queue=FCFS_B[2]: START of job 2.
 t=4.0, queue=FCFS_B[2]: ARRIVAL of job 4.
 t=5.0, queue=FCFS_B[2]: ARRIVAL of job 5.
 t=5.0, queue=FCFS_B[2]: DROP of job 5.
 t=6.0, queue=FCFS_B[2]: ARRIVAL of job 6.
 t=6.0, queue=FCFS_B[2]: DROP of job 6.
 t=7.0, queue=FCFS_B[2]: ARRIVAL of job 7.
 t=7.0, queue=FCFS_B[2]: DROP of job 7.
 t=7.6000000000000005, queue=FCFS_B[2]: DEPARTURE of job 2.
 t=7.6000000000000005, queue=FCFS_B[2]: START of job 3.
 t=8.0, queue=FCFS_B[2]: ARRIVAL of job 8.
 t=9.0, queue=FCFS_B[2]: ARRIVAL of job 9.
 t=9.0, queue=FCFS_B[2]: DROP of job 9.
 t=14.200000000000001, queue=FCFS_B[2]: DEPARTURE of job 3.
 t=14.200000000000001, queue=FCFS_B[2]: START of job 4.
 t=23.0, queue=FCFS_B[2]: DEPARTURE of job 4.
 t=23.0, queue=FCFS_B[2]: START of job 8.
 t=40.6, queue=FCFS_B[2]: DEPARTURE of job 8.

\end{lstlisting}
\end{lstfloat}

If, on the other hand, your {\em only\/} interest is in the
  fundamental \lstinline|RESET|, \lstinline|UPDATE| and \lstinline|STATE CHANGED|
  notifications,
  you can register a \lstinline|StdOutSimEntityListener|
  as shown in Listing \ref{simExample1_listener2_1}
  or, simpler but equivalent, Listing \ref{simExample1_listener2_2},
  and their corresponding output in Listing \ref{simExample1_listener2_out}.

\begin{lstfloat}
\begin{lstlisting}[
  caption={Creating and registering a \texttt{StdOutSimEntityListener}.},
  label=simExample1_listener2_1,
  basicstyle=\tiny]

  final SimEntityListener listener = new StdOutSimEntityListener ();
  queue.registerSimEntityListener (listener);

\end{lstlisting}
\end{lstfloat}


\begin{lstfloat}
\begin{lstlisting}[
  caption={Using \texttt{registerStdOutSimEntityListener}.},
  label=simExample1_listener2_2,
  basicstyle=\tiny]

  queue.registerStdOutSimEntityListener ();

\end{lstlisting}
\end{lstfloat}

\begin{lstfloat}
\begin{lstlisting}[
  caption={Example output of Listing \ref{simExample1_main} with the modified listener of Listings
  \ref{simExample1_listener2_1} or \ref{simExample1_listener2_2}},
  label=simExample1_listener2_out,
  basicstyle=\tiny]

StdOutSimEntityListener t=0.0, entity=FCFS_B[2]: UPDATE.
StdOutSimEntityListener t=0.0, entity=FCFS_B[2]: STATE CHANGED:
  => ARRIVAL [Arr[0]@FCFS_B[2]]
  => START [Start[0]@FCFS_B[2]]
  => DEPARTURE [Dep[0]@FCFS_B[2]]
StdOutSimEntityListener t=1.0, entity=FCFS_B[2]: UPDATE.
StdOutSimEntityListener t=1.0, entity=FCFS_B[2]: STATE CHANGED:
  => ARRIVAL [Arr[1]@FCFS_B[2]]
  => START [Start[1]@FCFS_B[2]]
  => STA_FALSE [StartArmed[false]@FCFS_B[2]]
StdOutSimEntityListener t=2.0, entity=FCFS_B[2]: UPDATE.
StdOutSimEntityListener t=2.0, entity=FCFS_B[2]: STATE CHANGED:
  => ARRIVAL [Arr[2]@FCFS_B[2]]
StdOutSimEntityListener t=3.0, entity=FCFS_B[2]: UPDATE.
StdOutSimEntityListener t=3.0, entity=FCFS_B[2]: STATE CHANGED:
  => ARRIVAL [Arr[3]@FCFS_B[2]]
StdOutSimEntityListener t=3.2, entity=FCFS_B[2]: UPDATE.
StdOutSimEntityListener t=3.2, entity=FCFS_B[2]: STATE CHANGED:
  => DEPARTURE [Dep[1]@FCFS_B[2]]
  => START [Start[2]@FCFS_B[2]]
StdOutSimEntityListener t=4.0, entity=FCFS_B[2]: UPDATE.
StdOutSimEntityListener t=4.0, entity=FCFS_B[2]: STATE CHANGED:
  => ARRIVAL [Arr[4]@FCFS_B[2]]
StdOutSimEntityListener t=5.0, entity=FCFS_B[2]: UPDATE.
StdOutSimEntityListener t=5.0, entity=FCFS_B[2]: STATE CHANGED:
  => ARRIVAL [Arr[5]@FCFS_B[2]]
  => DROP [Drop[5]@FCFS_B[2]]
StdOutSimEntityListener t=6.0, entity=FCFS_B[2]: UPDATE.
StdOutSimEntityListener t=6.0, entity=FCFS_B[2]: STATE CHANGED:
  => ARRIVAL [Arr[6]@FCFS_B[2]]
  => DROP [Drop[6]@FCFS_B[2]]
StdOutSimEntityListener t=7.0, entity=FCFS_B[2]: UPDATE.
StdOutSimEntityListener t=7.0, entity=FCFS_B[2]: STATE CHANGED:
  => ARRIVAL [Arr[7]@FCFS_B[2]]
  => DROP [Drop[7]@FCFS_B[2]]
StdOutSimEntityListener t=7.6000000000000005, entity=FCFS_B[2]: UPDATE.
StdOutSimEntityListener t=7.6000000000000005, entity=FCFS_B[2]: STATE CHANGED:
  => DEPARTURE [Dep[2]@FCFS_B[2]]
  => START [Start[3]@FCFS_B[2]]
StdOutSimEntityListener t=8.0, entity=FCFS_B[2]: UPDATE.
StdOutSimEntityListener t=8.0, entity=FCFS_B[2]: STATE CHANGED:
  => ARRIVAL [Arr[8]@FCFS_B[2]]
StdOutSimEntityListener t=9.0, entity=FCFS_B[2]: UPDATE.
StdOutSimEntityListener t=9.0, entity=FCFS_B[2]: STATE CHANGED:
  => ARRIVAL [Arr[9]@FCFS_B[2]]
  => DROP [Drop[9]@FCFS_B[2]]
StdOutSimEntityListener t=14.200000000000001, entity=FCFS_B[2]: UPDATE.
StdOutSimEntityListener t=14.200000000000001, entity=FCFS_B[2]: STATE CHANGED:
  => DEPARTURE [Dep[3]@FCFS_B[2]]
  => START [Start[4]@FCFS_B[2]]
StdOutSimEntityListener t=23.0, entity=FCFS_B[2]: UPDATE.
StdOutSimEntityListener t=23.0, entity=FCFS_B[2]: STATE CHANGED:
  => DEPARTURE [Dep[4]@FCFS_B[2]]
  => START [Start[8]@FCFS_B[2]]
StdOutSimEntityListener t=40.6, entity=FCFS_B[2]: UPDATE.
StdOutSimEntityListener t=40.6, entity=FCFS_B[2]: STATE CHANGED:
  => DEPARTURE [Dep[8]@FCFS_B[2]]
  => STA_TRUE [StartArmed[true]@FCFS_B[2]]

\end{lstlisting}
\end{lstfloat}

In most practical cases,
  you will need a listener that does a bit more than reporting to \lstinline|System.out|.
Of course, you can override the methods in \lstinline|StdOutSimQueueListener|
  to fit your purposes, but a better way is to use a \lstinline|DefaultSimQueueListener|,
  or, if you just want to process the fundamental notification (\lstinline|RESET|,
  \lstinline|UPDATE| and \lstinline|STATE CHANGED|),
  a \lstinline|DefaultSimEntityListener|.
Both listener type are concrete,
  but all required method implementations are empty.
In our next example,
  we take a \lstinline|DefaultSimQueueListener|
  and modify it in order to calculate the average
  job {\em sojourn\/} time.
This time,
  we create a separate \lstinline|class| named \lstinline|JobSojournTimeListener|
  for the listener,
  shown in Listing \ref{simExample1_avgSojournTime_class}.

\begin{lstfloat}
\begin{lstlisting}[
  caption={A (somewhat naive) queue listener that calculates the average job sojourn time.},
  label=simExample1_avgSojournTime_class,
  basicstyle=\tiny]

public class JobSojournTimeListener
extends DefaultSimQueueListener
{

  private final Map<SimJob, Double> jobArrTimes = new HashMap<> ();

  private int jobsPassed = 0;
  private double cumJobSojournTime = 0;
  
  @Override
  public void notifyArrival (double time, SimJob job, SimQueue queue)
  {
    if (this.jobArrTimes.containsKey (job))
      throw new IllegalStateException ();
    this.jobArrTimes.put (job, time);
  }
  
  @Override
  public void notifyDeparture (double time, SimJob job, SimQueue queue)
  {
    if (! this.jobArrTimes.containsKey (job))
      throw new IllegalStateException ();
    final double jobSojournTime = time - this.jobArrTimes.get (job);
    if (jobSojournTime < 0)
      throw new IllegalStateException ();
    this.jobArrTimes.remove (job);
    this.jobsPassed++;
    this.cumJobSojournTime += jobSojournTime;
  }

  @Override
  public void notifyDrop (double time, SimJob job, SimQueue queue)
  {
    notifyDeparture (time, job, queue);
  }
  
  public double getAvgSojournTime ()
  {
    if (this.jobsPassed == 0)
      return Double.NaN;
    return this.cumJobSojournTime / this.jobsPassed;
  }
  
}

\end{lstlisting}
\end{lstfloat}

In the class, we override the (courtesy) notifications for job arrivals,
  departures and drops.
When a job arrives, its arrival time is put into a private \lstinline|Map| (\lstinline|jobArrTimes|)
  for later reference.
When a job departs or is dropped,
  we retrieve its arrival time,
  calculate its sojourn time,
  and add the result to the cumulative sum
  of sojourn times, \lstinline|cumJobSojournTime|.
In order to later interpret this value,
  we also have to maintain the number of jobs passed
  in a private field \lstinline|jobsPassed|.
At any time,
  the class provides the average sojourn time (over all jobs {\em passed\/})
  through its \lstinline|getAvgSojournTime| method.
The calculation involved is trivial;
  the method returns \lstinline|Double.NaN| when
  no jobs have passed.

The use of \lstinline|JobSojournTimeListener| and the corresponding output
  are shown in Listings \ref{simExample1_avgSojournTime_main}
  and \ref{simExample1_avgSojournTime_out}, respectively.

\begin{lstfloat}
\begin{lstlisting}[
  caption={Our \texttt{FCFS\_B} example
           with the custom \texttt{JobSojournTimeListener}.},
  label=simExample1_avgSojournTime_main,
  basicstyle=\tiny]

    final SimEventList el = new DefaultSimEventList ();
    final int bufferSize = 2;
    final FCFS_B queue = new FCFS_B (el, bufferSize);
    final JobSojournTimeListener listener = new JobSojournTimeListener ();
    queue.registerSimEntityListener (listener);
    for (int j = 0; j < 10; j++)
    {
      final double jobServiceTime = (double) 2.2 * j;
      final double jobArrivalTime = (double) j;
      final String jobName = Integer.toString (j);
      final SimJob job = new DefaultSimJob (null, jobName, jobServiceTime);
      SimJQEventScheduler.scheduleJobArrival (job, queue, jobArrivalTime);
    }
    el.run ();
    System.out.println ("Average job sojourn time is " + listener.getAvgSojournTime () + ".");

\end{lstlisting}
\end{lstfloat}

\begin{lstfloat}
\begin{lstlisting}[
  caption={The output of Listing \ref{simExample1_avgSojournTime_main}.},
  label=simExample1_avgSojournTime_out,
  basicstyle=\tiny]

Average job sojourn time is 7.06.

\end{lstlisting}
\end{lstfloat}

Before going into details on the average sojourn time reported,
  we want to stress that our implementation of
  \lstinline|JobSojournTimeListener| is far from complete and even erroneous,
  although it works correctly in this specific (use) case.
For instance, it fails to consider the fact that jobs may
  {\em not\/} leave the queue (in whatever way).
Such jobs are named {\em sticky\/} jobs,
  and in a true application we would have to consider them.
Second,
  apart from \lstinline|DROP| and \lstinline|DEPARTURE|,
  there are other means by which a job
  can depart the queueing system
  (in particular, {\em revocations\/}).
%   see Section \ref{sec:guided:revocations}).
Third,
  the listener ignores \lstinline|RESET|
  notifications from the queue;
  a critical error as we shall see later.
%  as will become
%  clear later in Section \ref{sec:guided:reset}.
We will not further discuss these and other complications here,
  because our primary intention is to show you
  the mechanisms for creating and modifying listeners.
We just want to point out that the design of {\em robust\/}
  statistical listeners is more complicated that shown here.
We provide a thorough treatment in Chapter \ref{chap:statistics}.

Returning to the reported average job sojourn time.
Is it correct?
Well, in order to verify, we have no choice but to carefully analyze
  the behavior of the \lstinline|FCFS_B| queue under the given
  workload of jobs, as given in Table \ref{simExample1_analysis}.
The table shows for each job its
  job number (Job),
  arival time (Arr),
  required service time (ReQ),
  jobs waiting upon its arrival (WQA),
  start time (Start, if applicable),
  exit time (either due to departure or due to dropping),
  sojourn time (exit time minus arrival time),
  and remarks if applicable.
The final rows
  show the sum (TOT) and the average (AVG)
  of the required service times and
  the sojourn times,
  thus validating the result.

\begin{table}[h]
\caption{Analysis of job sojourn times in Listing \ref{simExample1_main}.}
\label{simExample1_analysis}
\begin{center}
\begin{tabular}{|l|l|r|l|r|r|r|l|}
\hline
Job & Arr & ReqS & WQA & Start & Exit & Sojourn & Remark \\
\hline
\hline
0 & 0.0 &  0.0 & $\{      \}$ & 0.0  &  0.0 &  0.0 & Exits immediately. \\ \hline
1 & 1.0 &  2.2 & $\{      \}$ & 1.0  &  3.2 &  2.2 &                    \\ \hline
2 & 2.0 &  4.4 & $\{      \}$ & 3.2  &  7.6 &  5.6 &                    \\ \hline
3 & 3.0 &  6.6 & $\{ 2    \}$ & 7.6  & 14.2 & 11.2 &                    \\ \hline
4 & 4.0 &  8.8 & $\{ 3    \}$ & 14.2 & 23.0 & 19.0 &                    \\ \hline
5 & 5.0 & 11.0 & $\{ 3, 4 \}$ & -    &  5.0 &  0.0 & Dropped.           \\ \hline
6 & 6.0 & 13.2 & $\{ 3, 4 \}$ & -    &  6.0 &  0.0 & Dropped.           \\ \hline
7 & 7.0 & 15.4 & $\{ 3, 4 \}$ & -    &  7.0 &  0.0 & Dropped.           \\ \hline
8 & 8.0 & 17.6 & $\{ 4    \}$ & 23.0 & 40.6 & 32.6 &                    \\ \hline
9 & 9.0 & 19.8 & $\{ 4, 8 \}$ & -    &  9.0 &  0.0 & Dropped.           \\ \hline
\hline
TOT   & & 99.0 &              &      &      & 70.6 &                    \\ \hline
\hline
AVG   & &  9.9 &              &      &      & 7.06 &                    \\ \hline
\hline
\end{tabular}
\end{center}
\end{table}

