In this book, we describe two \lstinline|Java|
  libraries for discrete-event simulation of queueing systems, (1) \lstinline|jsimulation|
  provides support for events, event lists, event scheduling and actions, and (2) \lstinline|jqueue| which
  extends \lstinline|jsimulation| with support
  for the simulation of queueing systems.
In Chapter \ref{chap:hello-world}
  we provide a guided tour
  to both packages,
  whereas in Chapters
  \ref{chap:events-actions-event-list}
  and
  \ref{chap:entities}
  \ref{chap:jobs-and-queues}
  we describe in detail
  \lstinline|jsimulation|
  and \lstinline|jqueues|,
  respectively.
In Chapters \ref{chap:fundamental-queues}
  and \ref{chap:composite-queues} we describe
  all available implementations
  of fundamental and composite
  queueing systems,
  respectively.
In Chapter \ref{chap:listeners} we {\bf XXX}.
Finally, in Chapter \ref{chap:statistics},
  we explain how to obtain statistics
  from a simulation (run).

Although we have at this time of writing not yet
  completed the road map for our next release,
  we intend to include the following new features:
\begin{itemize}
  \item The seeding of random-number generators
          upon a reset,
          either from a manually provided
          {\em seed value\/},
          or from a suitable {\em seed generator}.
        The latter, though, requires a study
          on the implementation of
          \lstinline|Java|'s \lstinline|Random|
          class,
          i.e.,
          its pseudo-random generator.
  \item A wider support for random-number generation,
          allowing other implementations than
          \lstinline|Java|'s \lstinline|Random| class.
  \item A new high-performance model
          for composite queues,
          either as a backwards-compatible
          extension to the present \lstinline|SimQueueComposite| interface,
          or as a new queue type alongside it.
        The main idea is to allow jobs to
          visit sub-queues themselves,
          instead of through delegate jobs.
        We actually tried to include this
          feature into the current release,
          but this led to an impossibly complex
          concept of composite queues,
          and we did not manage to properly
          implement the current set of
          composite queues (and their associated tests).
        Therefore, we reverted our efforts,
          and satisfied ourselves with the current
          model for composite queues.
  \item Several single-server processor-sharing queueing disciplines
          in which the relative execution rates of the jobs in the
          service area depends on either:
          \begin{itemize}
            \item The per-job QoS value, which we then need to restrict to a number.
                  Our improvised name for this queueing system is
                    \lstinline|DPS|, or {\em Discriminatory Processor Sharing}.
            \item The relative {\em share\/} of jobs with {\em identical\/}
                    QoS values, i.e., (sorry for the confusion here) the per-job-{\em class\/}
                    QoS value.
                  This approach requires a \lstinline|Map<P, Double>|,
                    mapping QoS values (of generic type \lstinline|P|)
                    onto relative {\em per-job\/} execution rates.
                  If applicable, each job present of a specific QoS value
                    (i.e., class of jobs) receives the same relative priority.
                  We intend to name this queueing system \lstinline|PPS|,
                    or {\em Priority Processor Sharing}.
            \item The relative {\em share\/} of the job {\em class}.
                  This approach is comparable to the previous one,
                    yet makes multiple jobs of a single job class
                    {\em share equally\/} their execution rate;
                    it also requires a \lstinline|Map<P, Double>|,
                    mapping QoS values (of generic type \lstinline|P|)
                    onto relative {\em per-class\/} execution rates.
                  We intend to name this queueing system \lstinline|GPPS|,
                    or {\em Group-Priority Processor Sharing}.
          \end{itemize}
        More background on these queueing systems,
          and some motivation for their naming is
          given in \cite{deJo2002}.
  \item The \lstinline|DUPS| (fundamental) queue type;
          the abbreviation stands for
          {\em Decay-Usage Processor Sharing}.
        In \lstinline|DUPS|,
          jobs are served on a single processor-sharing queue,
          but their relative execution rates
          are inversely proportional
          to their obtained service times.
  \item Support for servers with different "capacity".
        For instance, we are considering the
          development of a \lstinline|EncCap| encapsulator
          (composite) queue with an
          essential property \lstinline|Capacity|
          that "scales" the required service time on the
          encapsulated queue.
        We are also considering an extension to the
          \lstinline|SimJob| interface to
          include assessment of the required job service time
          in the presence of queues with varying capacity.
        At the moment, though, we do not consider
          server capacities that can {\em change\/} in time.
        (Since that would make the capacity a state property,
           and far more complicated.)
        Note that for certain queue types like
          server-less queues or \lstinline|SUR|,
          the concept of server capacity has no meaning.
        So we also need a good definition of the whole concept,
          including for which queue types it applies.
  \item Support for replications and batched means.
\end{itemize}
