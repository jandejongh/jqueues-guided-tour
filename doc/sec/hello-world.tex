In this section,
  we introduce our "Hello World" application for
  \lstinline|jqueues|\footnote{
  	In this Chapter,
  	whenever we refer to \texttt{jqueues},
  	we silently assume that \texttt{jsimulation}
  	is installed as well.},
  consisting of a \lstinline|FCFS| queue
  subject to arrivals of jobs with varying required service times.

In order to perform a simulation study in \lstinline|jqueues|,
  the following actions need to be taken:
\begin{itemize}
\item The creation of an event list;
\item The construction of one or more queues attached to the event list;
\item The selection of the method for listening to the queue(s);
\item The creation of a workload consisting of jobs
        and appropriately scheduling it onto the event list;
\item The execution of the event list;
\item The interpretation of the results,
        typically from the listener output.
\end{itemize}
Without much further ado,
  we show our "Hello World" example
  in Figure \ref{simExample1_main}.
We first create a single event list of type \lstinline|DefaultSimEventList|
  and a \lstinline|FCFS| queue attached to the event list
  (by virtue of the argument of \lstinline|FCFS|'s constructor).
On the queue, we register a newly created
  \lstinline|StdOutSimEntityListener|,
  issuing notifications to the standard output.
Note that queues and jobs are so-called {\em entities\/};
  these are the relevant objects with state subject to event invocation.
Subsequently,
  we create ten jobs named "0", "1", "2", $\ldots$,
  scheduled for arrival at the queue
  at $t=0$, $t=1$, $t=2$, $\ldots$,
  respectively,
  and set their respective service times.
We then schedule each job arrival on the event list.
Finally, we "run" the event list, i.e.,
  let it process the arrivals.

\begin{lstfloat}
\begin{lstlisting}[
  caption={A simple simulation with a single FCFS queue and ten jobs.},
  label=simExample1_main,
  basicstyle=\tiny]

  final SimEventList el = new DefaultSimEventList (0);
  final SimQueue queue = new FCFS (el);
  queue.registerSimEntityListener (new StdOutSimEntityListener ());
  for (int j = 0; j < 10; j++)
  {
    final double jobServiceTime = (double) 2.2 * j;
    final double jobArrivalTime = (double) j;
    final String jobName = Integer.toString (j);
    final SimJob job = new DefaultSimJob (null, jobName, jobServiceTime);
    SimJQEventScheduler.scheduleJobArrival (job, queue, jobArrivalTime);
  }
  el.run ();

\end{lstlisting}
\end{lstfloat}

The event list type
  \lstinline|DefaultSimEventList|
  will suffice for almost all practical cases,
  but it is essential to note already that
  a {\em single\/} event-list instance is typically used
  throughout {\em any\/} simulation program.
Its purpose of the event list is to hold scheduled {\em events\/}
  in non-decreasing order of {\em schedule time},
  and, upon request (in this case through \lstinline|el.run|),
  starts processing the scheduled events in sequence,
  invoking their associated {\em actions}.
In this case,
  the use of events remains hidden,
  because jobs are scheduled through the use of utility method
  \lstinline|scheduleJobArrival|.
The zero argument to the constructor denotes
  the simulation start time.
If you leave it out, the start time defaults to $-\infty$.

Our queue of choice is First-Come First-Served (FCFS).
The constructor takes the event list \lstinline|el|
  as argument.
The queueing system consists of a queue with infinite
  places to hold jobs,
  and a single server that "serves" the jobs in the queue in order
  of their arrival.
Once a queue has finished serving the (single) job,
  the job {\em departs\/} from the system.

So how long does it take to serve a job?
Well, in \lstinline|jqueues|,
  the default behavior is that
  a queue requests the job for its
  {\em required service time}.
In the particular case of \lstinline|DefaultSimJob|
  (there are many more job types),
  we provide a fixed service time (at {\em any\/} queue)
  upon creation through the third argument of the constructor.
  
The first argument of the \lstinline-DefaultSimJob-
  is the event list to which it is to be attached.
For jobs (well, at least the ones derived from \lstinline|DefaultSimJob|),
  it is often safe to set this to \lstinline|null|,
  although we could have equally well set it to \lstinline|el|.
However, {\em queues must always be attached to the event list\/};
  a \lstinline|null| value upon construction will throw an exception.

The (approximate) output of the code fragment
  of Listing \ref{simExample1_main}
  is shown in Listing \ref{simExample1_out} below.
Remarkably,
  the listing only shows two types of notifications,
  viz.,
  \lstinline|UPDATE|
  and
  \lstinline|STATE_CHANGED|,
  the latter of which can hold
  multiple "sub"-notifications.
Each notification outputs
  the name of the listener,
  the time on the event list,
  the queue (entity) that issues the notification,
  the notification's actual "major" type
  (\lstinline|UPDATE| or \lstinline|STATE_CHANGED|)
  and, if present,
  the sub-notifications.

\begin{lstfloat}
\begin{lstlisting}[
  caption={Example output of Listing \ref{simExample1_main}.},
  label=simExample1_out,
  basicstyle=\tiny]

StdOutSimEntityListener t=0.0, entity=FCFS: STATE CHANGED:
  => ARRIVAL [Arr[0]@FCFS]
  => START [Start[0]@FCFS]
  => DEPARTURE [Dep[0]@FCFS]
StdOutSimEntityListener t=1.0, entity=FCFS: UPDATE.
StdOutSimEntityListener t=1.0, entity=FCFS: STATE CHANGED:
  => ARRIVAL [Arr[1]@FCFS]
  => START [Start[1]@FCFS]
  => STA_FALSE [StartArmed[false]@FCFS]
StdOutSimEntityListener t=2.0, entity=FCFS: UPDATE.
StdOutSimEntityListener t=2.0, entity=FCFS: STATE CHANGED:
  => ARRIVAL [Arr[2]@FCFS]
StdOutSimEntityListener t=3.0, entity=FCFS: UPDATE.
StdOutSimEntityListener t=3.0, entity=FCFS: STATE CHANGED:
  => ARRIVAL [Arr[3]@FCFS]
StdOutSimEntityListener t=3.2, entity=FCFS: UPDATE.
StdOutSimEntityListener t=3.2, entity=FCFS: STATE CHANGED:
  => DEPARTURE [Dep[1]@FCFS]
  => START [Start[2]@FCFS]
StdOutSimEntityListener t=4.0, entity=FCFS: UPDATE.
StdOutSimEntityListener t=4.0, entity=FCFS: STATE CHANGED:
  => ARRIVAL [Arr[4]@FCFS]
StdOutSimEntityListener t=5.0, entity=FCFS: UPDATE.
StdOutSimEntityListener t=5.0, entity=FCFS: STATE CHANGED:
  => ARRIVAL [Arr[5]@FCFS]
StdOutSimEntityListener t=6.0, entity=FCFS: UPDATE.
StdOutSimEntityListener t=6.0, entity=FCFS: STATE CHANGED:
  => ARRIVAL [Arr[6]@FCFS]
StdOutSimEntityListener t=7.0, entity=FCFS: UPDATE.
StdOutSimEntityListener t=7.0, entity=FCFS: STATE CHANGED:
  => ARRIVAL [Arr[7]@FCFS]
StdOutSimEntityListener t=7.6000000000000005, entity=FCFS: UPDATE.
StdOutSimEntityListener t=7.6000000000000005, entity=FCFS: STATE CHANGED:
  => DEPARTURE [Dep[2]@FCFS]
  => START [Start[3]@FCFS]
StdOutSimEntityListener t=8.0, entity=FCFS: UPDATE.
StdOutSimEntityListener t=8.0, entity=FCFS: STATE CHANGED:
  => ARRIVAL [Arr[8]@FCFS]
StdOutSimEntityListener t=9.0, entity=FCFS: UPDATE.
StdOutSimEntityListener t=9.0, entity=FCFS: STATE CHANGED:
  => ARRIVAL [Arr[9]@FCFS]
StdOutSimEntityListener t=14.200000000000001, entity=FCFS: UPDATE.
StdOutSimEntityListener t=14.200000000000001, entity=FCFS: STATE CHANGED:
  => DEPARTURE [Dep[3]@FCFS]
  => START [Start[4]@FCFS]
StdOutSimEntityListener t=23.0, entity=FCFS: UPDATE.
StdOutSimEntityListener t=23.0, entity=FCFS: STATE CHANGED:
  => DEPARTURE [Dep[4]@FCFS]
  => START [Start[5]@FCFS]
StdOutSimEntityListener t=34.0, entity=FCFS: UPDATE.
StdOutSimEntityListener t=34.0, entity=FCFS: STATE CHANGED:
  => DEPARTURE [Dep[5]@FCFS]
  => START [Start[6]@FCFS]
StdOutSimEntityListener t=47.2, entity=FCFS: UPDATE.
StdOutSimEntityListener t=47.2, entity=FCFS: STATE CHANGED:
  => DEPARTURE [Dep[6]@FCFS]
  => START [Start[7]@FCFS]
StdOutSimEntityListener t=62.60000000000001, entity=FCFS: UPDATE.
StdOutSimEntityListener t=62.60000000000001, entity=FCFS: STATE CHANGED:
  => DEPARTURE [Dep[7]@FCFS]
  => START [Start[8]@FCFS]
StdOutSimEntityListener t=80.20000000000002, entity=FCFS: UPDATE.
StdOutSimEntityListener t=80.20000000000002, entity=FCFS: STATE CHANGED:
  => DEPARTURE [Dep[8]@FCFS]
  => START [Start[9]@FCFS]
StdOutSimEntityListener t=100.00000000000001, entity=FCFS: UPDATE.
StdOutSimEntityListener t=100.00000000000001, entity=FCFS: STATE CHANGED:
  => DEPARTURE [Dep[9]@FCFS]
  => STA_TRUE [StartArmed[true]@FCFS]

\end{lstlisting}
\end{lstfloat}

Apart from the \lstinline-STATE CHANGED-, \lstinline-UPDATE-
  and \lstinline-START_ARMED- lines in the output,
  the notifications pretty much speak for themselves.
We even get notified when jobs start service (\lstinline-START-).
The \lstinline-START_ARMED- notifications
  refer to state changes in a special \lstinline|boolean| attribute of a
  queue named its \lstinline-StartArmed- property.
Since you will hardly need it in practical applications,
  we will not delve into it,
  but it is crucial for the implementation
  of certain more complex (composite) queueing systems.
Suffice it to say that the \lstinline|StartArmed| property
  {\em in this particular case\/}
  signals whether the queue is idle.

The two top-level notification types,
  \lstinline|UPDATE| and \lstinline|STATE CHANGED|
  are essential.
Upon every change to a queue's state,
  the queue is obliged to issue the fundamental \lstinline|STATE CHANGED| notification,
  exposing the queue's new state (including its notion of time).
The \lstinline|UPDATE| notification
  has the same function,
  but it is fired {\em before\/} any changes have been applied,
  thus revealing the queue's {\em old\/} state,
  including the time at which the old state was obtained.
Hence,
  every \lstinline|STATE CHANGED| notification
  {\em must\/} be preceded with an
  \lstinline|UPDATE| notification.
The \lstinline|UPDATE| notification is
  crucial for the implementation of statistics
  (among others).
  
The use of \lstinline|STATE_CHANGED| notifications may
  appear strange at first sight
  as many other implementations would
  report each of the sub-notifications individually.
However, an important aspect of a queue's contract is that
  {\em it must report state changes atomically
       in order to meet queue invariants}.
This means that listeners,
  when notified,
  will always see the queue in a consistent state,
  i.e., in a state that respects the invariant(s).
This is one of the (we think) most distinguishing features of
  \lstinline|jqueues|.
Going back to our example: An important invariant of FCFS
  and many other queueing systems
  is that there cannot be jobs
  waiting in queue while the server is idle.
It is easy to see that individual notifications for
  \lstinline|ARRIVAL| and \lstinline|START| would
  lead to violations of this invariant:
  Suppose that a job arrives at an idle FCFS queue.
  Using individual notifications,
  the queue has no other option than to
  issue a \lstinline|ARRIVAL|
  notification
  immediately followed by a \lstinline|START|.
In between both, the queue would expose a state
  that is inconsistent with the invariant
  because the server is idle (the job has not started yet),
  while there is a job in its waiting queue.
Note that another invariant of FCFS is that
  it cannot be serving jobs with zero required service time.
This explains the arrival, start and departure sub-notifications
  for job $0$ are all in a single atomic \lstinline|STATE_CHANGED|.
  
This concludes our "Hello World" example.
There is obviously a lot more to tell,
  but the good news is that our example
  has already revealed the most important concepts
  of \lstinline|jqueues| like
  the event list, events, entities, queues, jobs, listeners
  and notifications.
The remaining complexity is in the richness and variation of these
  basic concepts.