This document describes the
  publicly available
  \lstinline|jsimulation|
  and \lstinline|jqueues|
  \lstinline|Java|
  libraries for
  discrete-event simulation of queueing systems.
Personally,
  I (merely) scratched the surface of queueing theory
  at Twente University back in the eighties,
  while working on my Master's Thesis
  on an operating system for {\em transputers}.
Transputers are fast RISC processors
  with multiple on-chip communication links;
  back then, they were envisioned to become the building blocks
  of future massively parallel computer systems.
Since our main applications of interest were in robotics,
  I attempted basic queueing theory in an attempt to find
  hard real-time response-time guarantees,
  in order to meet physical-world, mostly safety-related, deadlines.

During the largest part of the nineties,
  I worked on my PhD at Delft University of Technology.
This time,
  I got to study queueing systems modeling
  {\em distributed computing systems},
  which by then had overtaken parallel systems
  in terms of scientific interest.
The main purpose of the research was to devise
  and analyze scheduling strategies for
  dividing in space and in time
  the computing resources of a
  (closed) distributed system among groups of users,
  according to predefined policies (named
  {\em share scheduling\/} at that time).
In order to gain quantitative insight,
  I used the classic \lstinline|DEMOS|
  (Discrete Event Modeling On Simula) software
  running on the \lstinline|SIMULA|
  programming language.
I made several modifications and extensions to the software,
  in order for it to suit my needs.
For instance,
  it lacked support for so-called
  {\em processor-sharing\/}
  queueing disciplines in which
  a server ("processor") distributes
  at any time its service capacity
  among (a subset of) jobs present.
In addition,
  I needed a non-standard set of statistics
  gathered from the simulation runs.
In the end,
  both \lstinline|DEMOS| and \lstinline|SIMULA| itself
  proved flexible enough to study the research questions.

Like \lstinline|DEMOS|,
  the \lstinline|jsimulation|
  and \lstinline|jqueues|
  \lstinline|Java|
  software packages described in this book
  feature discrete-event simulation
  of queueing systems.
The libraries are, as a combo,
  somewhat comparable to
  the \lstinline|DEMOS|,
  yet there are important differences nonetheless.
For instance, the libraries
  focus exclusively at {\em algorithmic\/}
  modeling of queueing systems and job visits;
  they do not cover additionally required features
  like sophisticated random-number generation,
  probability distributions,
  gathering and analyzing statistics,
  and sophisticated reporting;
  features all integrated in \lstinline|DEMOS|.
In that sense, \lstinline|DEMOS|
  is a more complete package.
On the other hand,
  the packages feature a larger range of
  queueing-system types,
  and, for instance,
  a model for constructing new queueing systems
  through {\em composition\/} of
  other queues.
In addition,
  much care has been put into the
  {\em atomicity\/} of certain events,
  which allows for a wider range of
  {\em queue invariants\/} supported.
Despite these differences,
  \lstinline|DEMOS| has been a major inspiration
  in the design and implementation of
  \lstinline|jsimulation|
  and \lstinline|jqueues|.

For my current employer, TNO,
  I have performed, over the past decade
  (or even decades?),
  many simulation studies in \lstinline|Java|
  related to the vehicle-to-vehicle communications
  in (future) Intelligent Transportation Systems,
  studying, for instance,
  position dissemination over CSMA/CA wireless networks
  for Cooperative Adaptive Cruise Control
  and Platooning.
At some point,
  I realized that it would be feasible to
  extract some useful and stable
  \lstinline|Java| libraries from my ever increasing
  software repositories,
  and release them into the public domain.
So, in a way,
  the libraries can be considered "collateral damage" from a variety of projects.

% I have dedicated the software and this documentation,
%   with gratitude,
%   to the designer of \lstinline|DEMOS|,
%   Graham Birtwistle
%   and, in memory,
%   to the designers of \lstinline|SIMULA|,
%   Ole-Johan Dahl and Kristen Nygaard.