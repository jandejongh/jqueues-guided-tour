In this chapter,
  we introduce {\em all\/}
  \lstinline|SimQueue| implementations
  for {\em fundamental\/}
  queueing systems in \lstinline|jqueues|
  Release 5.
A queueing system is named {\em fundamental\/} if
  it is not constructed by composition
  of other queueing system instances
  as in for instance
  tandem, feedback and Jackson queueing systems.
(This is a \texttt{jqueues}-specific definition.)
Such {\em composite\/} queueing systems
  are introduced and described in detail in
  the next chapter.
In the present chapter,
  we describe the queueing systems to the extent
  that they can be put to use in a simulation.
It does not aim to be complete
   in the individual descriptions, however.
  
\section{Serverless Queues}

In so-called {\em serverless\/} queues,
  jobs never enter the service area,
  in other words,
  {\em they never start}.
As a result,
  the service area of a serverless queue is
  always empty,
  server-access credits are ignored,
  and the \lstinline|startArmed|
  property is always \lstinline|false|.
In Table \ref{tab:serverless-queues},
  we list all serverless queues implemented in
  \lstinline|jqueues|, accompanied with a brief
  description.

\begin{table}[!htbp]
	\label{tab:serverless-queues}
	\caption{The serverless queueing systems
             in \texttt{jqueues} Release 5
             at a glance.}
	\begin{center}
		\begin{tabular}{|l|l|}
			\hline
			\multicolumn{2}{|c|}{} \\
			\multicolumn{2}{|c|}{Serverless Queues} \\
			\multicolumn{2}{|c|}{} \\
			\hline
            \hline
			\lstinline|DROP|
              &
              {\em Drops jobs immediately upon arrival.}
              \\ \hline
			\lstinline|SINK|
              &
              {\em Lets jobs wait indefinitely.}
              \\ \hline
			\lstinline|ZERO|
              &
              {\em Lets jobs depart immediately upon arrival.}
              \\ \hline
			\lstinline|DELAY|
              &
              {\em Lets jobs depart after a fixed wait time.}
              \\ \hline
			\lstinline|GATE|
              &
              {\em Lets jobs depart immediately}
              \\
              &
              {\em while the queue has "gate-passage credits"}.
              \\ \hline
			\lstinline|ALIMIT|
              &
              {\em Drops (admitted) jobs if needed to not exceed} 
              \\
              &
              {\em a given arrival-rate limit.} 
              \\ \hline
			\lstinline|DLIMIT|
              &
              {\em Lets jobs wait if needed to not exceed} 
              \\
              &
              {\em a given departure-rate limit.} 
              \\ \hline
			\lstinline|LeakyBucket|
              &
              {\em Lets jobs wait if needed to not exceed} 
              \\
              &
              {\em a given departure-rate limit;} 
              \\
              &
              {\em the waiting area has finite size, though.} 
              \\ \hline
			\lstinline|WUR|
              &
              {\em Lets jobs wait until
                   the next (admitted) arrival.} 
              \\ \hline
		\end{tabular}
	\end{center}
\end{table}

It is important to realize that despite the virtual
  absence of the service area,
  a serverless queue still reports
  running out of and regaining
  server-access credits,
  conform Section \ref{sec:guided:sac}.
In addition,
  all queues support queue-access vacations,
  as outlined in Section \ref{sec:guided:qav},
  and job revocations
  described in Section \ref{sec:revoke}.
  
The \lstinline|DROP|,
  \lstinline|SINK|,
  \lstinline|ZERO|
  and \lstinline|WUR|
  queueing systems
  hopefully
  speak for themselves.
These do not have additional properties,
  operations, or sub-notification types
  in addition to the bare \lstinline|SimQueue|
  interface.

The \lstinline|DELAY| queue
  lets jobs depart (from the waiting area)
  a fixed time after their arrival,
  captured in the queue's
  \lstinline|waitTime|
  essential property
  with type \lstinline|double|.
Setting the property value to zero
  turns \lstinline|DELAY| into
  a \lstinline|ZERO|;
  setting it to infinity
  (\lstinline|Double.POSITIVE_INFINITY|)
  turns it into a \lstinline|SINK|.
  
The \lstinline|GATE| queueing system is a bit
  more involved.
It features a \lstinline|gatePassgeCredits|
  state property of type \lstinline|int|
  holding the remaining number of jobs
  allowed to depart
  in order of arrival.
On each departure,
  the property value is decremented,
  and if it reaches zero,
  jobs are no longer allowed to depart;
  they must reside in the waiting area instead.
However,
  the value \lstinline|Integer.MAX_VALUE|,
  the reset value,
  is treated as infinity;
  prohibiting the decrements upon departures.
If the property value reaches zero,
  a \lstinline|GATE_CLOSED| sub-notification
  is issued;
  if credits are regained a \lstinline|GATE_OPEN|.
The property value can be set through the
  external \lstinline|SetGatePassageCredits|
  operation.
Note the high degree of similarity
  with a \lstinline|SimQueue|'s
  server-access credits
  with \lstinline|GATE|'s
  gate-passage credits.
Informally,
  the latter replaces the former
  in view of the lack of starting jobs.
  
The \lstinline|ALIMIT|,
  \lstinline|DLIMIT|
  and \lstinline|LeakyBucket| queues
  impose limits on arrival (for \lstinline|ALIMIT|)
  or departure (for \lstinline|DLIMIT|
  and \lstinline|LeakyBucket|) rates.
Each is equipped
  with a \lstinline|rateLimited|
  state property of \lstinline|boolean|
  type indicating that
  the queue is currently in
  {\em rate limitation},
  as well as a \lstinline|rateLimit|
  essential property of type
  \lstinline|double|.
In the current release,
  changes to the \lstinline|rateLimited|
  state property value are
  {\em not\/} reported through
  dedicated sub-notification types,
  but since this is a violation
  of the \lstinline|SimQueue| contract
  (unless we think of the property as "internal"),
  this will be fixed in a future release.

The \lstinline|ALIMIT| queue
  drops arriving jobs while it is
  in rate limitation,
  whereas both
  \lstinline|DLIMIT|
  and \lstinline|LeakyBucket|
  put arriving jobs into the waiting area
  until the rate limitation has expired.
(This is in fact accomplished by an
  anonymous internal operation.)
The only difference between the latter two
  is that \lstinline|LeakyBucket|
  has a waiting area {\em of finite size\/}
  as mandated by its \lstinline|bufferSize|
  essential property.
It is of type \lstinline|int|,
  and, as usual,
  the value \lstinline|Integer.MAX_VALUE|
  is treated as infinity.
The \lstinline|LeakyBucket| queue
  drop jobs upon arrival
  if it is in rate limitation
  {\em and\/} there are
  already \lstinline|bufferSize|
  other jobs present in the waiting area.
  
For \lstinline|ALIMIT|,
  rate limitation starts upon an admitted arrival\footnote{
  In this context, an arriving job that is not dropped
  due to queue-access vacation {\em or\/}
  due to rate limitation},
  and always ends $1/\mbox{\lstinline|rateLimit|}$
  after that.
As a special case,
  setting \lstinline|rateLimit| to zero
  effectively turns \lstinline|ALIMIT|
  into a \lstinline|DROP|;
  setting it to infinity
  turns \lstinline|ALIMIT| into a \lstinline|ZERO|.
Note that whatever the property value,
  the \lstinline|ALIMIT|
  queue {\em never\/} has jobs in its
  waiting area (or its visit area, for that matter).
For \lstinline|DLIMIT| and \lstinline|LeakyBucket|,
  rate limitation {\em always\/} starts
  or continues at a departure;
  it lasts for $1/\mbox{\lstinline|rateLimit|}$,
  {\em if\/} at the projected end time
  the waiting area is empty.
Otherwise,
  the job longest in the waiting area
  departs,
  restarting rate limitation
  for another $1/\mbox{\lstinline|rateLimit|}$.
  
Finally,
  a noteworthy disambiguation
  of \lstinline|LeakyBucket|
  with zero \lstinline|bufferSize|
  is that, shortly put,
  absence of rate limitation takes precedence
  over the absent buffer space.
This means that {\em if\/} \lstinline|LeakyBucket|
  is {\em not\/} in rate limitation,
  arriving jobs will depart immediately
  (starting a new rate-limitation period).

\section{Non-Preemptive Queues}
\label{sec:non-preemptive-queues}

In \lstinline|jqueues|,
  so-called {\em non-preemptive\/}
  queues
  serve jobs in their service area
  until they depart.
Jobs {\em never\/} depart from the waiting area.
Non-preemptive queues in \lstinline|jqueues|
  differ in
  (1) the size of the waiting area (\lstinline|bufferSize|),
  (2) the number of (always equal) servers in the service area
      (\lstinline|numberOfServers|),
  (3) the order in which jobs are selected
      for service from the waiting area
  and/or
  (4) the selection policy for dropping jobs
      in case the number of jobs waiting
      is about to exceed the buffer size.
In Table \ref{tab:non-preemptive-queues},
  we list all non-preemptive queues implemented in
  \lstinline|jqueues|, accompanied with a brief
  description.
The queues are presented in several groups,
  each with important common structure and behavior.
  
\begin{table}[!htbp]
	\label{tab:non-preemptive-queues}
	\caption{The non-preemptive queueing systems
             in \texttt{jqueues} Release 5
             at a glance.}
	\begin{center}
		\begin{tabular}{|l|l|}
			\hline
			\multicolumn{2}{|c|}{} \\
			\multicolumn{2}{|c|}{Non-Preemptive Queues} \\
			\multicolumn{2}{|c|}{} \\
			\hline
            \hline
              &
              {\em Serves jobs
                   in order of arrival
                   until completion}
              \\
              &
              {\em as defined by the jobs'
                   requested service times...}
              \\
              &
              \\
			\lstinline|FCFS|
              &
              {\em - with a single server
                     and infinite buffer size.}
              \\
			\lstinline|FCFS_B|
              &
              {\em - with a single server
                     and finite buffer size.}
              \\
			\lstinline|FCFS_c|
              &
              {\em - with multiple servers
                     and infinite buffer size.}
              \\
			\lstinline|FCFS_B_c|
              &
              {\em - with multiple servers
                     and finite buffer size.}
              \\
			\lstinline|NoBuffer_c|
              &
              {\em - with multiple servers
                     and no buffer at all.}
              \\
			\lstinline|IS|
              &
              {\em - with infinite servers
                     and infinite buffer size.}
              \\
            \hline
              &
              {\em Serves jobs
                   in reverse order of arrival
                   until completion}
              \\
              &
              {\em as defined by the jobs'
                   requested service times...}
              \\
              &
              \\
			\lstinline|LCFS|
              &
              {\em - with a single server
                     and infinite buffer size.}
              \\
			\lstinline|LCFS_B|
              &
              {\em - with a single server
                     and finite buffer size.}
              \\
            \hline
              &
              {\em Serves jobs in order of arrival...}
              \\
              &
              \\
			\lstinline|IC|
              &
              {\em - in zero service time
                     with infinite buffer size.} 
              \\
			\lstinline|IS_CST|
              &
              {\em - in constant service time
                     with infinite buffer size.}
              \\
            \hline
              &
              {\em Serves jobs
                   until completion
                   with a single server}
              \\
              &
              {\em as defined by the jobs'
                   requested service times...}
              \\
              &
              \\
			\lstinline|SJF|
              &
              {\em - in order
                     of increasing requested service time...} 
              \\
			\lstinline|LJF|
              &
              {\em - in order
                     of decreasing requested service time...} 
              \\
			\lstinline|RANDOM|
              &
              {\em - in random order...}
              \\
              &
              \\
              &
              {\em with infinite buffer size.} 
              \\
            \hline
			\lstinline|SUR|
              &
              {\em Serves jobs until
                   the next (admitted) arrival} 
              \\
              &
              {\em with a single server
                   and infinite buffer size.} 
              \\
            \hline
		\end{tabular}
	\end{center}
\end{table}

Before going into details on the respective groups
  of non-preemptive queueing systems,
  it is important to realize that each of them
  is subject to the mechanism
  of server-access credits explained in
  Section \ref{sec:guided:sac}.
Also, each has at least two essential properties:
  (1) \lstinline|bufferSize|
  limits the number of jobs in the waiting area
  and
  (2)
  \lstinline|numberOfServers| limits the number
  of jobs in the service area.
Both are of type \lstinline|int|,
  and as usual,
  \lstinline|Integer.MAX_VALUE|
  is interpreted as infinity.

The center of gravity,
  i.e.,
  the most general form,
  of the first group is
  the \lstinline|FCFS_B_c|
  queueing system,
  featuring user-settable
  \lstinline|bufferSize|
  (\lstinline|B|)
  and
  \lstinline|numberOfServers|
  (\lstinline|c|)
  essential properties.
All other members in this group
  are specializations of \lstinline|FCFS_B_c|
  fixing one or both properties.
In \lstinline|FCFS_B_c|,
  in absence of queue-access vacations,
  an arriving job is taking into service
  {\em immediately}
  (1) the queue has non-zero server-access credits, {\em and\/}
  (2) the queue has strictly fewer jobs in the service area
      than the value of the \lstinline|numberOfServers|
      property (respecting the convention that
      \lstinline|Integer_MAX_VALUE| represents infinity).
Otherwise,
  the arriving job is entered into
  the waiting area of the queue (at the tail),
  provided that
  the current number of jobs in the waiting area
  is strictly smaller that the \lstinline|bufferSize|
  property.
If this is not the case,
  the arriving job is dropped.
  
In \lstinline|FCFS_B_c|,
  the other epochs at which jobs are eligible to start
  are (1) the granting of server-access credits,
  (2) the scheduled departure of a job (from the service area)
  or (3) the revocation of a job from the service area.
In any of these cases,
  if there are server-access credits,
  servers available
  (respecting the case of an infinite number of servers)
  and jobs in the waiting area,
  the job in the waiting area with the earliest
  arrival time is started,
  i.e.,
  admitted to the service area.
Once admitted to the service area,
  a job is served until completion as prescribed by
  its requested service time,
  see Section \ref{sec:requested-service-time},
  unless it is somehow revoked.
  
The second group of non-preemptive queues
  listed in Table \ref{tab:non-preemptive-queues},
  centered around \lstinline|LCFS_B|,
  differs from the first group in only two ways:
\begin{itemize}
  \item In case of depletion of buffer space,
          the job that arrived {\em earliest\/}
          is dropped (instead of the arriving jobs).
  \item The service area selects the job
          that arrived latest as the eligible job to
          start.
\end{itemize}

The remaining groups
  in Table \ref{tab:non-preemptive-queues}
  speak for themselves;
  since each queue has infinite buffer space,
  we do not have to worry about job drops.
The only things different
  from the previous groups
  are the selection of jobs for admission
  to the service area
  or the time required
  to serve a job
  once admitted
  to the service area.
  
\section{Preemptive Queues}
\label{sec:preemptive-queues}

Unlike non-preemptive queueing systems
  described in the previous section,
  preemptive queueing systems
  do not always provide service to
  jobs in the service area,
  and may in fact interrupt ({\em preempt})
  execution of a job in favor of another job
  (in the service area).
In Table \ref{tab:preemptive-queues},
  we list all preemptive queues implemented in
  \lstinline|jqueues|, accompanied with a brief
  description.

\begin{table}[!htbp]
	\label{tab:preemptive-queues}
	\caption{The preemptive queueing systems
             in \texttt{jqueues} Release 5
             at a glance.}
	\begin{center}
		\begin{tabular}{|l|l|}
			\hline
			\multicolumn{2}{|c|}{} \\
			\multicolumn{2}{|c|}{Preemptive Queues} \\
			\multicolumn{2}{|c|}{} \\
			\hline
            \hline
              &
              {\em Serves jobs until completion,}
              \\
              &
              {\em as defined by the jobs'
                   requested service times,}
              \\
              &
              {\em or until preemption}
              \\
              &
              {\em with a single server
                   and infinite buffer size...}
              \\
              &
              \\
			\lstinline|P_LCFS|
              &
              {\em - in reverse arrival order.}
              \\
			\lstinline|SRTF|
              &
              {\em - in increasing order of
                     remaining service time.}
              \\
            \hline
		\end{tabular}
	\end{center}
\end{table}

At the moment a job is preempted at a server,
  one of several things can happen,
  as determined by the \lstinline|preemptionStrategy|
  essential property.
In \lstinline|jqueues| Release 5,
  the preemption strategy is of type \lstinline|enum|
  named \lstinline|PreemptionStrategy|,
  and it can take the following values:
\begin{itemize}
  \item \lstinline|DROP|: The preempted job is dropped.
  \item \lstinline|RESUME|: The preempted job is put on hold,
                              after calculating the remaining service time.
                            Future service resumption continues
                              at the point where the previous service
                              was interrupted.
                            (This is the default.) 
  \item \lstinline|RESTART|: The preempted job is put on hold.
                             Future service resumption
                               requires the job to be served from scratch.
  \item \lstinline|REDRAW|: The preempted job is put on hold.
                            Future service resumption
                              requires the job to be served from scratch
                              with a new required service time.
                            (This is not supported in \lstinline|jqueues| Release 5.)
  \item \lstinline|DEPART|: The preempted job departs,
                              even though it may not have finished
                              its service requirements.
  \item \lstinline|CUSTOM|: A different strategy that those mentioned
                              above is applied to preempted jobs.
                            The strategy needs to be specified
                            in the concrete \lstinline|SimQueue| implementation.
\end{itemize}
Not all preemptive-queue implementations
  support all preemption strategies.
For instance, the \lstinline|REDRAW| strategy
  is not supported by any implementation,
  because it conflicts with the contract
  that the requested service time of a \lstinline|SimJob|
  has to remain constant
  during a (single) \lstinline|SimQueue| visit.

The preemption strategy
  is a parameter that can be passed upon construction
  to all implemented preemptive queues
  in \lstinline|jqueues|.
Its default value is \lstinline|RESUME|.
If you pass an unsupported preemption strategy,
  the queue will throw an \lstinline|UnsupportedOperationException|.

For consistency with
  non-preemptive queueing systems,
  both \lstinline|P_LCFS| and \lstinline|SRTF|
  support the
  \lstinline|bufferSize|
  and \lstinline|numberOfServers|
  essential properties,
  but their value are fixed
  at
  \lstinline|Integer.MAX_VALUE|,
  representing infinity,
  and unity,
  respectively.
  
\section{Processor-Sharing Queues}

In the preceding two sections
  on non-preemptive and preemptive
  queueing systems, respectively,
  a common characteristic was that
  the service area consists of
  a countable number of servers,
  each serving at most one job,
  but not necessarily {\em all job\/}
  in the service area (in particular,
  this was found not to be true in general
  for preemptive queues).
In so-called {\em processor-sharing\/}
  queueing systems,
  most of this model still holds,
  with the exception that
  the a single server is capable of providing
  service to {\em multiple\/} jobs simultaneously,
  distributing its total service capacity
  among them.
They are of extreme theoretical importance in queueing theory,
  because they often can be regarded as the limiting case
  of "time sharing": Serving a single job for small amounts of time,
  then saving its state
  and resuming service to another job from its saved state.
In Table \ref{tab:processor-sharing-queues},
  we list all processor-sharing queues implemented in
  \lstinline|jqueues|, accompanied with a brief
  description.
  
\begin{table}[!htbp]
	\label{tab:processor-sharing-queues}
	\caption{The processor-sharing queueing systems
             in \texttt{jqueues} Release 5
             at a glance.}
	\begin{center}
		\begin{tabular}{|l|l|}
			\hline
			\multicolumn{2}{|c|}{} \\
			\multicolumn{2}{|c|}{Processor-Sharing Queues} \\
			\multicolumn{2}{|c|}{} \\
			\hline
            \hline
              &
              {\em Serves simultaneously with a single server and infinite buffer size,}
              \\
              &
              {\em distributing its capacity among jobs in the service area...}
              \\
              &
              \\
			\lstinline|PS|
              &
              {\em - at equal rates.}
              \\
			\lstinline|CUPS|
              &
              {\em - with the lowest obtained service time at equal rates.}
              \\
			\lstinline|SocPS|
              &
              {\em - at rates linearly proportional to their remaining service times.}
              \\
            \hline
		\end{tabular}
	\end{center}
\end{table}

The three processor-sharing queues
  have in common that they consist
  of a single server in the service area,
  and that in presence of a server-access credit,
  and arriving job is admitted immediately to
  the service area.
The three systems only differ in the way
  the server distributes its capacity
  among jobs in the service area.

The \lstinline|PS| implements
  the (Egalitarian) Processor Sharing
  queueing discipline
  with a single server.
It is probably well-known to many readers.
In \lstinline|PS|,
  the server simply distributes its capacity equally
  among {\em all\/} jobs in the service area.

In \lstinline|CUPS| (Catch-Up Processor Sharing),
  the server distributes its capacity equally
  among all jobs in the service area with the
  lowest {\em obtained\/} service time.
Whenever a job starts,
  it enters the service area
  with zero obtained service time,
  and assuming there are no new arrivals after that,
  it will therefore be served exclusively.
The jobs that were served a priori,
  will have to wait in the service area
  until the new job has reached sufficient
  service to "catch up"
  and join that set.
From that point on,
  the server will distribute its capacity among
  the "a priori set of jobs" augmented with the
  newly arrived job.
The resulting set of jobs
  will be served at equal rates
  until they catch up with the next set of jobs,
  etc.
In case a job has received sufficient
  service as mandated by its requested service time,
  it will leave the set of "jobs in service",
  and if that set turns empty because of that,
  \lstinline|CUPS| will switch to the next
  set of jobs with minimum obtained service time,
  if any.
If not,
  the server will turn idle.
Intuitively,
  \lstinline|CUPS| gives extreme preferential
  treatment to jobs with small requested service time.
To our knowledge,
  the \lstinline|CUPS| queueing system has not been
  described in queueing-system literature before\footnote{
  But we have not fully checked this,
  and would be highly interested in relevant references.}.

In \lstinline|SocPS| (Social Processor Sharing),
  the single server distributes its capacity
  among jobs in the service area
  at rate linearly proportional to their
  {\em remaining service time}.
This implies that,
  ignoring for instance revocations,
  {\em all jobs in the service area will depart at the same time.}
This includes jobs with zero requested service time;
  they will simply not be served if other jobs
  with non-zero remaining service times are present
  in the service area.
As \lstinline|CUPS|, \lstinline|SocPS| is
  a queueing discipline we have not found earlier
  in literature.
  
For consistency with
  non-preemptive and preemptive queueing systems
  described earlier,
  the processor-sharing implementations support the
  \lstinline|bufferSize|
  and \lstinline|numberOfServers|
  essential properties,
  but their value are fixed
  at
  \lstinline|Integer.MAX_VALUE|,
  representing infinity,
  and unity,
  respectively.
  
\section{QoS Queues}
\label{sec:qos-queues}

We end this chapter with a separate presentation of
  QoS queueing systems,
  even though in \lstinline|jqueues|
  each of the QoS queueing systems
  could be categorized in one of the earlier
  described classes (non-preemptive, preemptive
  and processor-sharing queueing systems)
  as well.
Recall from Section \ref{sec:qos}
  that in such systems,
  visiting jobs are treated differently
  corresponding to their membership of
  some "group" of jobs\footnote{
    We avoid the more common "job class" terminology
    since it is highly confusing in the \lstinline|jqueues|
    context.}.
To that purpose,
  jobs and queues need
  non-\lstinline|null| and compatible
  values for their \lstinline|qosClass|
  property;
  in addition,
  each job needs a value
  for its \lstinline|qosValue|
  property, indicating the "group"
  of jobs to which it belongs.
In Table \ref{tab:qos-queues},
  we list all QoS queues implemented in
  \lstinline|jqueues|, accompanied with a brief
  description.
  
\begin{table}[!htbp]
	\label{tab:qos-queues}
	\caption{The QoS queueing systems
             in \texttt{jqueues} Release 5
             at a glance.}
	\begin{center}
		\begin{tabular}{|l|l|}
			\hline
			\multicolumn{2}{|c|}{} \\
			\multicolumn{2}{|c|}{QoS Queues} \\
			\multicolumn{2}{|c|}{} \\
			\hline
            \hline
			\lstinline|HOL|
              &
              {\em Serves jobs until completion
                   in order of increasing \lstinline|QoS| values}
              \\
              &
              {\em with a single server and infinite buffer size.}
              \\
              &
              {\em Serves jobs identical \lstinline|QoS|
                   values in arrival order.}
              \\
              &
              {\em The server can serve at most one job at a time;}
              \\
              &
              {\em it does not preempt jobs.}
              \\
              &
              \\
			\lstinline|PQ|
              &
              {\em Serves jobs until completion or preemption}
              \\
              &
              {\em in order of increasing \lstinline|QoS| values}
              \\
              &
              {\em with a single server and infinite buffer size.}
              \\
              &
              {\em Preempts a job in service upon the start of another job}
              \\
              &
              {\em with strictly smaller \lstinline|QoS| value.}
              \\
              &
              {\em Serves jobs with identical \lstinline|QoS|
                   values in arrival order.}
              \\
              &
              {\em The server can serve at most one job at a time.}
              \\
              &
              \\
			\lstinline|HOL_PS|
              &
              {\em Serves simultaneously (equally distributing its capacity)}
              \\
              &
              {\em all earliest-arrived jobs with distinct \lstinline|QoS| values}
              \\
              &
              {\em with a single server and infinite buffer size.}
              \\
              &
              {\em Serves jobs with identical \lstinline|QoS|
                   values in arrival order.}
              \\
            \hline
		\end{tabular}
	\end{center}
\end{table}

From their brief description,
  one deduces that \lstinline|HOL|
  is a non-preemptive queueing system,
  and that \lstinline|PQ|
  and \lstinline|HOL_PS|
  are preemptive and processor-sharing ones,
  respectively.
For \lstinline|HOL| and \lstinline|PQ|,
  the \lstinline|qosClass| used must support
  (at least) partial ordering of the values used.
Therefore,
  the \lstinline|qosClass| must be \lstinline|Comparable|
  in the \lstinline|Java| sense.
For \lstinline|HOL_PS|, this is not required,
  any \lstinline|class|
  or \lstinline|interface| will do;
  the \lstinline|qosValue| property values
  are compared using the \lstinline|Object.equals|
  method,
  following the convention in \lstinline|Java|'s
  \lstinline|Set| interface in the
  \lstinline|Collections| framework.
  
Since \lstinline|PQ| is a preemptive queueing system,
  it supports the preemption strategy mechanism
  introduced in Section \ref{sec:preemptive-queues}.